\documentclass[11pt]{article}
\usepackage[margin=1.0in]{geometry}

\usepackage[T1]{fontenc}

%%%
%%%  Page Setup
%%%
\special{papersize=8.5in,11in}
\setlength{\pdfpagewidth}{8.5in}
\setlength{\pdfpageheight}{11in}

%%% Figure caption declaration
\usepackage[labelfont=bf]{caption}
%%% Spacing
\addtolength{\parskip}{0.5ex}


%%%
%%%  Lists
%%%
\usepackage{enumitem}
\setlist{itemsep=0.5ex,parsep=0pt}             % more compact lists

%%%
\usepackage{fancyhdr}
\usepackage{xspace}
%\usepackage{enumerate}
\usepackage{verbatim}
\usepackage{url}
\usepackage{latexsym,amsthm,amsmath,amsfonts,amssymb,stmaryrd,mathtools}

\usepackage[usenames]{color}
\usepackage{graphicx}

\usepackage{listings}
\usepackage{wrapfig}


%%%
%%%  Tables
%%%
\usepackage{booktabs}
%\usepackage{color}
\usepackage[usenames,dvipsnames]{xcolor}
\usepackage{colortbl}
\usepackage{multirow}
\usepackage{supertabular}
%%%


\usepackage[colorlinks,
citecolor=Sepia,
linkcolor=Blue,
pagebackref=true
%pagebackref=true      turn this on and off for citation back references.
]{hyperref}%         % for online version

\usepackage{natbib}
\usepackage{url}

%\usepackage{times}
\usepackage{txfonts}

%% My stuff
%%%%%%%%%%%%%%%%%%%%%%%%%%%%%%%%%%%%%%%%%%%%%%%%%%%%%%%%%%%%%%%%%%%%%%
%% mac.tex
%%
%% Umut A. Acar
%% Macros for adaptive computation paper.
%%%%%%%%%%%%%%%%%%%%%%%%%%%%%%%%%%%%%%%%%%%%%%%%%%%%%%%%%%%%%%%%%%%%%%

\newcommand{\readarrow}{\ensuremath{\Longrightarrow}}
\newcommand{\currenttime}{\ttt{currentTime}\xspace}
\newcommand{\tags}[1]{\ensuremath{\mathsf{tags}(#1)}}
\newcommand{\weight}[1]{\ensuremath{\mathsf{weight}(#1)}}
\newcommand{\dist}[2]{\ensuremath{\delta(#1,#2)}}


\newcommand{\qqquad}{\quad\quad\quad}

\newcommand{\cutspace}{\vspace{-4mm}}

\newcommand{\bomb}[1]{\fbox{\mbox{\emph{\bf {#1}}}}}

\newcommand{\myparagraph}[1]{{\bf {#1}}}
\newcommand{\vparagraph}[1]{\vspace{-.1in}\paragraph{#1}}

% formatting stuff
\newcommand{\codecolsep}{1ex}



\newcommand{\rlabel}[1]{\hspace*{-1mm}\mbox{\small{\bf ({#1})}}}

\newcommand{\tablerow}{\\[5ex]}
\newcommand{\tableroww}{\\[7ex]}
\newcommand{\tableline}{
\vspace*{2ex}\\
\hline\\ 
\vspace*{2ex}}

% Don't care
\newcommand{\dontcare}{\_
}

%% filter and quicksort stuff
\newcommand{\ncf}[2]{C^{fil}_{\ensuremath{{#1},{#2}}}}
\newcommand{\ncq}[1]{C^{qsort}_{\ensuremath{#1}}}
\newcommand{\nuf}[3]{P^{fil}_{#1,(\ensuremath{{#2},{#3}})}}
\newcommand{\nuq}[2]{P^{qsort}_{(\ensuremath{{#1},{#2}})}}

%% shorthands
\newcommand{\ddg}{{\sc ddg}}
\newcommand{\ncpa}{change-propagation algorithm}
\newcommand{\adg}{{\sc adg}}
\newcommand{\nwrite}{\texttt{write}}
\newcommand{\nread}{\texttt{read}}
\newcommand{\nmodr}{\texttt{mod}}
\newcommand{\ttt}[1]{\texttt{#1}}
\newcommand{\nmodl}{\texttt{modl}}
\newcommand{\nnil}{\ttt{NIL}}
\newcommand{\ncons}[2]{\ttt{CONS({\ensuremath{#1},\ensuremath{#2}})}}
\newcommand{\nfilter}{\texttt{filter}}
\newcommand{\nfilterp}{\texttt{filter'}}
\newcommand{\naqsort}{\texttt{qsort'}}
\newcommand{\nqsort}{\texttt{qsort}}
\newcommand{\nqsortp}{\texttt{qsort'}}
\newcommand{\nnewMod}{\texttt{newMod}}
\newcommand{\nchange}{\texttt{change}}
\newcommand{\npropagate}{\texttt{propagate}}
\newcommand{\ndest}{\texttt{d}}
\newcommand{\ninit}{\texttt{init}}



%% Comment sth out. 
\newcommand{\out}[1] {}
\newcommand{\sthat}{\ensuremath{~|~}}

%% definitions
\newcommand{\defi}[1]{{\bfseries\itshape #1}}


% Code listings.
\newcounter{codeLineCntr}
\newcommand{\codeLine}
 {\refstepcounter{codeLineCntr}{\thecodeLineCntr}}
\newcommand{\codeLineL}[1]
 {\refstepcounter{codeLineCntr}\label{#1}{\thecodeLineCntr}}
\newcommand{\codeLineNN}{} %% NN = No-Number (and no change to counter)
\newcommand{\codeNoLine}{}

\newenvironment{codeListing}
 {\setcounter{codeLineCntr}{0}
%  \fontsize{10}{12}
 % the first one is width the second is height
 \fontsize{9}{11}
  \fontsize{8}{8}
  \vspace{-.1in}
  \ttfamily\begin{tabbing}}
  {\end{tabbing}
   \vspace{-.1in}}

\newenvironment{codeListing8}
 {\setcounter{codeLineCntr}{0}
%  \fontsize{8}{10}
  \fontsize{8}{8}
  \vspace{-.1in}
  \ttfamily
  \begin{tabbing}}
 {\end{tabbing}
 \vspace{-.1in}
}

\newenvironment{codeListing8h}
 {\setcounter{codeLineCntr}{0}
  \fontsize{8.5}{10.5}
  \vspace{-.1in}
  \ttfamily
  \begin{tabbing}}
 {\end{tabbing}
 \vspace{-.1in}
}


\newenvironment{codeListing9}
 {\setcounter{codeLineCntr}{0}
  \fontsize{9}{11}
  \vspace{-.1in}
  \ttfamily
  \begin{tabbing}}
 {\end{tabbing}
 \vspace{-.1in}
}

\newenvironment{codeListing10}
 {\setcounter{codeLineCntr}{0}
  \fontsize{10}{12}
  \vspace{-.1in}
  \ttfamily
  \begin{tabbing}}
 {\end{tabbing}
 \vspace{-.1in}
}


\newenvironment{codeListingNormal}
 {\setcounter{codeLineCntr}{0}
  \vspace{-.1in}
  \ttfamily
  \begin{tabbing}}
 {\end{tabbing}
 \vspace{-.1in}
}

\newcommand{\codeFrame}[1]
{\begin{center}\fbox{\parbox[t]{\columnwidth}{#1}}\end{center}
% \vspace*{-.15in}
}

\newcommand{\halfBox}[1]
{\begin{center}\fbox{\parbox[t]{\columnwidth}{#1}}\end{center}
% \vspace*{-.15in}
}

\newcommand{\fullBox}[1]
{\begin{center}\fbox{\parbox[t]{\textwidth}{#1}}\end{center}
% \vspace*{-.15in}
}

%%Note this is redefined in local-mac.tex for each paper.
\newcommand{\fixedCodeFrame}[1]
{
\begin{center}
\fbox{
\parbox[t]{0.9\columnwidth}{
#1
}
}\end{center}
\vspace{-.2in}
}

% Footnote commands.
\newcommand{\footnotenonumber}[1]{{\def\thempfn{}\footnotetext{#1}}}

% Margin notes - use \notesfalse to turn off notes.
\setlength{\marginparwidth}{0.6in}
\reversemarginpar
\newif\ifnotes
\notestrue
\newcommand{\longnote}[1]{
  \ifnotes
    {\medskip\noindent Note:\marginpar[\hfill$\Longrightarrow$]
      {$\Longleftarrow$}{#1}\medskip}
  \fi}
\newcommand{\note}[1]{
  \ifnotes
    {\marginpar{\raggedright{\tiny #1}}}
  \fi}
\newcommand{\notered}[1]{\ifnotes
    {\marginpar{\raggedright{\tiny
          {\sf\color{red} #1}}}}
    \fi}

\newcommand{\mathcolor}[2]{\text{\textcolor{#1}{\ensuremath{#2}}}}

% Stuff not wanted.
\newcommand{\punt}[1]{}

% Sectioning commands.
\newcommand{\subsec}[1]{\subsection{\boldmath #1 \unboldmath}}
\newcommand{\subheading}[1]{\subsubsection*{#1}}
\newcommand{\subsubheading}[1]{\paragraph*{#1}}

% Reference shorthands.
\newcommand{\spref}[1]{Modified-Store Property~\ref{sp:#1}}
\newcommand{\prefs}[2]{Properties~\ref{p:#1} and~\ref{p:#2}}
\newcommand{\pref}[1]{Property~\ref{p:#1}}


\newcommand{\partref}[1]{Part~\ref{part:#1}}
\newcommand{\chref}[1]{Chapter~\ref{ch:#1}}
\newcommand{\chreftwo}[2]{Chapters \ref{ch:#1} and~\ref{ch:#2}}
\newcommand{\chrefthree}[3]{Chapters \ref{ch:#1}, and~\ref{ch:#2}, and~\ref{ch:#3}}
\newcommand{\secref}[1]{Section~\ref{sec:#1}}
\newcommand{\subsecref}[1]{Subsection~\ref{subsec:#1}}
\newcommand{\secreftwo}[2]{Sections \ref{sec:#1} and~\ref{sec:#2}}
\newcommand{\secrefthree}[3]{Sections \ref{sec:#1},~\ref{sec:#2},~and~\ref{sec:#3}}
\newcommand{\appref}[1]{Appendix~\ref{app:#1}}
\newcommand{\figref}[1]{Figure~\ref{fig:#1}}
\newcommand{\figreftwo}[2]{Figures \ref{fig:#1} and~\ref{fig:#2}}
\newcommand{\figrefthree}[3]{Figures \ref{fig:#1}, \ref{fig:#2} and~\ref{fig:#3}}
\newcommand{\figreffour}[4]{Figures \ref{fig:#1},~\ref{fig:#2},~\ref{fig:#3}~and~\ref{fig:#4}}
\newcommand{\figpageref}[1]{page~\pageref{fig:#1}}
\newcommand{\tabref}[1]{Table~\ref{tab:#1}}
\newcommand{\tabreftwo}[2]{Tables~\ref{tab:#1} and~\ref{tab:#1}}

\newcommand{\stref}[1]{step~\ref{step:#1}}
\newcommand{\caseref}[1]{case~\ref{case:#1}}
\newcommand{\lineref}[1]{line~\ref{line:#1}}
\newcommand{\linereftwo}[2]{lines \ref{line:#1} and~\ref{line:#2}}
\newcommand{\linerefthree}[3]{lines \ref{line:#1},~\ref{line:#2},~and~\ref{line:#3}}
\newcommand{\linerefrange}[2]{lines \ref{line:#1} through~\ref{line:#2}}
\newcommand{\thmref}[1]{Theorem~\ref{thm:#1}}
\newcommand{\thmreftwo}[2]{Theorems \ref{thm:#1} and~\ref{thm:#2}}
\newcommand{\thmpageref}[1]{page~\pageref{thm:#1}}
\newcommand{\lemref}[1]{Lemma~\ref{lem:#1}}
\newcommand{\lemreftwo}[2]{Lemmas \ref{lem:#1} and~\ref{lem:#2}}
\newcommand{\lemrefthree}[3]{Lemmas \ref{lem:#1},~\ref{lem:#2},~and~\ref{lem:#3}}
\newcommand{\lempageref}[1]{page~\pageref{lem:#1}}
\newcommand{\corref}[1]{Corollary~\ref{cor:#1}}
\newcommand{\defref}[1]{Definition~\ref{def:#1}}
\newcommand{\defreftwo}[2]{Definitions \ref{def:#1} and~\ref{def:#2}}
\newcommand{\defpageref}[1]{page~\pageref{def:#1}}
\renewcommand{\eqref}[1]{Equation~(\ref{eq:#1})}
\newcommand{\eqreftwo}[2]{Equations (\ref{eq:#1}) and~(\ref{eq:#2})}
\newcommand{\eqpageref}[1]{page~\pageref{eq:#1}}
\newcommand{\ineqref}[1]{Inequality~(\ref{ineq:#1})}
\newcommand{\ineqreftwo}[2]{Inequalities (\ref{ineq:#1}) and~(\ref{ineq:#2})}
\newcommand{\ineqpageref}[1]{page~\pageref{ineq:#1}}
\newcommand{\itemref}[1]{Item~\ref{item:#1}}
\newcommand{\itemreftwo}[2]{Item~\ref{item:#1} and~\ref{item:#2}}

% Useful shorthands.
\newcommand{\abs}[1]{\left| #1\right|}
\newcommand{\card}[1]{\left| #1\right|}
\newcommand{\norm}[1]{\left\| #1\right\|}
\newcommand{\floor}[1]{\left\lfloor #1 \right\rfloor}
\newcommand{\ceil}[1]{\left\lceil #1 \right\rceil}
  \renewcommand{\choose}[2]{{{#1}\atopwithdelims(){#2}}}
\newcommand{\ang}[1]{\langle#1\rangle}
\newcommand{\paren}[1]{\left(#1\right)}
\newcommand{\prob}[1]{\Pr\left\{ #1 \right\}}
\newcommand{\expect}[1]{\mathrm{E}\left[ #1 \right]}
\newcommand{\expectsq}[1]{\mathrm{E}^2\left[ #1 \right]}
\newcommand{\variance}[1]{\mathrm{Var}\left[ #1 \right]}
%\newcommand{\twodots}{\mathinner{\ldotp\ldotp}}

% Standard number sets.
\newcommand{\reals}{{\mathrm{I}\!\mathrm{R}}}
\newcommand{\integers}{\mathbf{Z}}
\newcommand{\naturals}{{\mathrm{I}\!\mathrm{N}}}
\newcommand{\rationals}{\mathbf{Q}}
\newcommand{\complex}{\mathbf{C}}

% Set notation
\newcommand{\sect}{\cap}

% Special styles.
%\newcommand{\proc}[1]{\ifmmode\mbox{\textsc{#1}}\else\textsc{#1}\fi}
\newcommand{\procdecl}[1]{
  \proc{#1}\vrule width0pt height0pt depth 7pt \relax}
%  \newcommand{\func}[1]{\ifmmode\mathrm{#1}\else\textrm{#1}fi} %
%  Multiple cases.  
\renewcommand{\cases}[1]{\left\{
  \begin{array}{ll}#1\end{array}\right.}
  \newcommand{\cif}[1]{\mbox{if $#1$}} 

%% spacing hacks
\newcommand{\longpage}{\enlargethispage{\baselineskip}}
\newcommand{\shortpage}{\enlargethispage{-\baselineskip}}



%% Notes, todos, and remarks
\newcounter{remark}[section]

\newcommand{\myremark}[3]{
\refstepcounter{remark}
\[
\left\{
\sf 
\parbox{0.8\columnwidth}
{
{\bf {#1}'s remark~\theremark:} 
{#3}
}
\right.
\]
\marginpar{\bf {#2}.~\theremark}
}



%\newcommand{\uremark}[1]{\myremark{Umut}{U}{#1}}
\newcommand{\uremark}[1]{}
\newcommand{\gremark}[1]{\myremark{Guy}{G}{#1}}
\newcommand{\mremark}[1]{\myremark{Mike}{M}{#1}}
%\newcommand{\ur}[1]{\uremark{#1}}
\newcommand{\ur}[1]{}
\newcommand{\gr}[1]{\gremark{#1}}
\newcommand{\mr}[1]{\mremark{#1}}


\newcommand{\todo}[1]{{\bf{[TODO:{#1}]}}}

\newcommand{\defeq}{\stackrel{\text{def}}{=}}
\newcommand{\setbuild}[2]{\{#1~|~#2\}}

%
\newcommand{\twork}[1]{\mathcal{W}\left({#1}\right)}
\newcommand{\tspan}[1]{\mathcal{S}\left({#1}\right)}
\newcommand{\tmax}[2]{\operatorname{Max}\left({#1},{#2}\right)}

\newcommand{\bigml}{\textsf{BigML}\xspace}

\newcommand{\taskcost}{I}
\newcommand{\taskgc}{III}
\newcommand{\taskhmm}{III}
\newcommand{\subtaskgcs}{II.A}
\newcommand{\subtaskgcp}{II.B}
\newcommand{\tasksch}{II}
\newcommand{\taskac}{IV}
\newcommand{\taskpml}{V}
\newcommand{\taskeval}{VI}

%% language
\newcommand{\lfs}{\ensuremath{\mathcal{L}_{FS}}\xspace}
\newcommand{\lfp}{\ensuremath{\mathcal{L}}\xspace}

\newcommand{\kwparcompose}{~~~~||~~~~}
\newcommand{\kwseqcompose}{~~~~;~~~~}
\newcommand{\cdparens}[1]{\mathcd{(}{#1}\mathcd{)}}
\newcommand{\kw}[1]{\mbox{\ttt{#1}}}
\newcommand{\Int}{\kw{int}}
\newcommand{\Letpar}{\kw{letpar}}
\newcommand{\Slet}{\kw{let}}
\newcommand{\Mod}[1]{{#1}~\kw{mod}}
\newcommand{\kwn}{\kw{n}}
\newcommand{\kwlet}[3]{\kw{let}~{#1}={#2}~\kw{in}~{#3}~\kw{end}}
\newcommand{\kwletpar}[5]{\kw{letpar}~\cdparens{{#1},{#3}}=\cdparens{{#2},{#4}}~\kw{in}~{#5}~\kw{end}}
\newcommand{\kwifthenelse}[3]{\kw{if}\,{#1}\,\kw{then}\,{#2}\,\kw{else}\,{#3}}
\newcommand{\kwfun}[3]{\ensuremath{\kw{fun}~{#1}~{#2}~\kw{is}~{#3}~\kw
{end}}}
\newcommand{\kwsfun}[3]{\ensuremath{\kw{fun}_{\kw{s}}~{#1}~{#2}~\kw{is}~{#3}~\kw{end}}}
\newcommand{\kwcfun}[3]{\ensuremath{\kw{fun}_{\kw{c}}~{#1}~{#2}~\kw{is}~{#3}~\kw{end}}}
\newcommand{\kwpair}[2]{\ensuremath{\lparen{#1},{#2}}\rparen}
 \newcommand{\kwpar}[2]{\ensuremath{\llparenthesis\,{#1},{#2}\,\rrparenthesis}}
\newcommand{\kwapply}[2]{\ensuremath{{#1}~{#2}}}
\newcommand{\kwnadd}[2]{\ensuremath{{#1} \oplus {#2}}}

\newcommand{\kwmod}[1]{\ensuremath{\kw{mod}\cdparens{#1}}}
\newcommand{\kwread}[3]{\ensuremath{\kw{read}~{#1}~\kw{as}~{#2}~\kw{in}~{#3}~\kw{end}}}
\newcommand{\kwfst}[1]{\ensuremath{\kw{fst}\cdparens{#1}}}
\newcommand{\kwsnd}[1]{\ensuremath{\kw{snd}\cdparens{#1}}}

\newcommand{\trread}[3]{\kw{READ}_{#1}{#2}~\kw{IN}~{#3}}
\newcommand{\trletpar}[3]{\kw{INPAR}~{#1}~\kw{AND}~{#2}~\kw{THEN}~{#3}}

%% shorthands
\renewcommand{\a}{\ensuremath{\alpha}}
\renewcommand{\b}{\ensuremath{\beta}}
\newcommand{\h}{\ensuremath{\eta}}
\renewcommand{\r}{\ensuremath{\rho}}
\newcommand{\s}{\ensuremath{\sigma}}



% Relations
\newcommand{\red}{\Downarrow}
\newcommand{\redgc}{\stackrel{gc?}{\Longrightarrow}}
\newcommand{\alloc}{\stackrel{alloc}\Longrightarrow}
\newcommand{\la}{\leftarrow}
\newcommand{\ra}{\rightarrow}

\newcommand{\sunion}[2]{{#1} \stackrel{?}{\bigcup} {#2}}
\newcommand{\spush}[2]{{#1} \stackrel{?}{\downarrow} {#2}}

% store, heap
\newcommand{\mks}[2]{{#1} \stackrel{?}{::} {#2}}
\newcommand{\mkh}[2]{{#1}_{#2}}
\newcommand{\mkhe}{\mkh{\emptyset}{\emptyset}}
\newcommand{\hemp}{\emptyset}

\newcommand{\locs}[1]{\mathcal{L}({#1})}
\newcommand{\htop}[1]{\mathcal{T}({#1})}
\newcommand{\roots}[1]{\mathcal{R}({#1})}


\newcommand{\hmm}{\textsf{HMM}}


% Syntax highlighting
\newdimen\zzlistingsize
\newdimen\zzlistingsizedefault
\zzlistingsizedefault=9pt
\newdimen\kwlistingsize
\kwlistingsize=9pt
\zzlistingsize=\zzlistingsizedefault
\gdef\lco{black}
\newcommand{\lcRed}{\gdef\lco{nred}}
\newcommand{\lcBlack}{\gdef\lco{black}}
\newcommand{\lcNormal}{\lcBlack}
\newcommand{\lcBlue}{\gdef\lco{nblue}}
\newcommand{\lcGreen}{\gdef\lco{ngreen}}
\newcommand{\lcPurple}{\gdef\lco{npurple}}
\newcommand{\Lstbasicstyle}{\fontsize{\zzlistingsize}{1.1\zzlistingsize}\ttfamily\color{\lco}}
%\newcommand{\keywordstyle}{\fontsize{0.9\zzlistingsize}{1.0\zzlistingsize}\bf}
%\newcommand{\keywordstyle}{\fontsize{\kwlistingsize}{1.1\kwlistingsize}\normalfont\bf\color{\lco}}
\newcommand{\keywordstyle}{\fontsize{1.09\kwlistingsize}{\kwlistingsize}\normalfont\bf\color{\lco}}
\newlength{\zzlstwidth}
\settowidth{\zzlstwidth}{{\Lstbasicstyle~}}
\newcommand{\lcm}{\color{\lco}}

\newcommand{\tts}{\texttt{\$S}}
\newcommand{\ttc}{\texttt{\$C}}

\lstset{
  basewidth=\zzlstwidth,
%  basicstyle=\Lstbasicstyle,
  basicstyle=\small\ttfamily,
%  identifierstyle=,
  numbers=left, numberstyle=\tiny, numbersep=5pt,
%  frame=single,
  keywordstyle=\keywordstyle,
  keywords={fun,cfun,sfun,read,case,of,type,functor,structure,signature,sig,struct,datatype,let,letpar,val,fn,in,end,as,select,mod,apply,sapply,capply,write},  %nil,cons,
  commentstyle=\rmfamily\slshape\color{\lco},
  morecomment=[n]{(*}{*)},
  lineskip={-0pt},
%  columns=[l]flexible,
  columns=[l]fullflexible,
  keepspaces=true,
  mathescape=true,
% NOTE: need TWO sets of braces around each definition below!
  literate={=>}{{$\lcm\Rightarrow$}}1 %
%           {fn}{{$\lambda$}}1 %
           {Mod}{{$\lcGreen\text{\Lstbasicstyle Mod}\lcNormal$}}3 %
           {Read}{{$\lcGreen\text{\Lstbasicstyle Read}\lcNormal$}}4 %
           {Write}{{$\lcGreen\text{\Lstbasicstyle Write}\lcNormal$}}5 %
           {sapply}{{$\lcm\text{\keywordstyle apply}^\S$}}6 %
           {capply}{{$\lcm\text{\keywordstyle apply}^\C$}}6 %
           {sfun}{{$\lcm\text{\keywordstyle fun}^\S$}}4 %
           {cfun}{{$\lcm\text{\keywordstyle fun}^\C$}}4 %
           {=\\S}{{$\lcm\text{\Lstbasicstyle=}\S$}}1
           {=\\C}{{$\lcm\text{\Lstbasicstyle=}\C$}}1
%           {^S}{{$\lcm^\S$}}1
%           {^C}{{$\lcm^\C$}}1
           {^S}{{$\lcm\;\,\tts$}}1
           {^C}{{$\lcm\;\,\ttc$}}1
           {^SS}{{$\lcm^{\S, \S}$}}1
           {^SC}{{$\lcm^{\S, \C}$}}1
           {^CS}{{$\lcm^{\C, \S}$}}1
           {^CC}{{$\lcm^{\C, \C}$}}1
           {^D}{{$\lcm^\delta$}}1
           {^Dh}{{$\lcm^{\delta_h}$}}1
           {^Dt}{{$\lcm^{\delta_t}$}}1
           {->}{{$\lcm\rightarrow$}}2
           {->^S}{{$\lcm\hspace{-1pt}\arrowsymbol{\S}\hspace{-1pt}$}}2
           {->^C}{{$\lcm\hspace{-1pt}\arrowsymbol{\C}\hspace{-1pt}$}}2
           {->^D}{{$\lcm\hspace{-1pt}\arrowsymbol{\delta}\hspace{-1pt}$}}2
           {->^Dh}{{$\lcm\hspace{-1pt}\arrowsymbol{\delta_h}\hspace{-1pt}$}}2
           {->^Dt}{{$\lcm\hspace{-1pt}\arrowsymbol{\delta_t}\hspace{-1pt}$}}2
%           {-c->}{${c\rightarrow}$}3
%           {-s->}{${s\rightarrow}$}3
%           {-c->}{$\rightarrow$}1
%           {-s->}{$\rightarrow$}1
%           {_}{{\lcm\_}}1
           {'a}{{$\alpha$}}1
           {'b}{{$\beta$}}1
           {'d}{{$\lcm\delta$}}1
}

%space left between floats.
\setlength{\floatsep}{2mm}
% space between last top float or first bottom float and the text.
\setlength{\textfloatsep}{2mm}
% space left on top and bottom of an in-text float.
\setlength{\intextsep}{2.5mm}
% space above caption
\setlength{\abovecaptionskip}{1mm}
%space below caption
\setlength{\belowcaptionskip}{1mm}

%% Change figure handling
\renewcommand{\topfraction}{0.80}
\renewcommand{\textfraction}{0.2}
\renewcommand{\floatpagefraction}{0.80}

\usepackage[small,compact]{titlesec} 
%\usepackage[compact]{titlesec} 


\usepackage{enumitem}
\setlist{itemsep=0.5ex,parsep=0pt}             % more compact lists

\titlespacing{\paragraph}{
  0pt}{             % left margin
  0.5\baselineskip}{% space before (vertical)
  1em}            % space after (horizontal)


\newcommand{\pt}[1]{\noindent
  {\color{blue}{\textbf{Point.}#1}}\smallskip}
\newcommand{\q}[1]{\noindent {\color{red}{\textbf{Question:} #1}}\smallskip}

%\date{}

%\title{SHF: Small: Functional Abstractions and Languages for Dynamic Big Data}


%%% this package has to be right before begin{doc}
\usepackage{microtype}

\title{Masters Projects}
\author{Umut A. Acar \and Arthur Chargueraud \and Mike Rainey}
\begin{document}

\maketitle

\section{Introduction}

This document describes several projects that can lead to interesting
research contributions and a proposed approach to conducting the
projects.

\section{Expectations and Reports.}

We expect each student to complete a master's report, consisting
probably of about 20-50 pages of text, typeset in LaTeX, that reports
the work performed, the results obtained, and puts them all in the
context of existing work on the topic.  Each master's report will also
come with an implementation completed in the C++ language using the
PASL library tools, which include a library for writing parallel
programs, as well as related tools for evaluating their performance.

We suggest the following timeline for each project.  This timeline
assumes a two-year project but it can be adjusted (stretched or
compressed) to 1 or 3 years based on each students constraints.  The
specifics will  be figured out for each project individually.

\begin{itemize}

\item {\bf Month 1: preliminary results.}  The student will complete a
  preliminary study of the problem and complete a 1 page report on
  what he or she has come up with. The preliminary results could be of
  the following nature: completing the development of an algorithm or
  a prototype that for example, works only in the sequential case.

\item {\bf Month 2: related work.}  The student will complete a review
  of the related work and complete a 2 page report discussing the
  related work and the limitation of the related work in relation to
  the proposed project.  There report will include an additional 1
  page material where the student discusses a technique or approach
  for addressing the limitations described above.

\item {\bf Month 3: results-1.}  The student will complete an initial
  result that demonstrates that the approach can work. This can
  consist of for example an algorithm and its implementation along
  with analysis and empirical results that show that the proposed
  approach works.  There will be a 2-page {\em result report}
  describing the results and a 1/2 page in the planned extensions.

\item {\bf Month 4: results-2.}  The student will build on the
  previous results and complete a 2.5-page result report as described
  above.

\item {\bf Month 5: results-3.}  The student will build on the
  previous results and complete a 2.5-page result report as described
  above.

\item {\bf Month 6: midterm report.}  The student will put together
  all the reports written together into a single 10-page document and
  wrote a one page plan about how to finish.


\item {\bf Month 9: completion plan.}  The student will finalize all
  results and implementation.  The code will be checked into the
  repository.  We will then pair each student with another {\em
    partner} student.  These students will swap their results and
  their code and each student will check the others work.  This will
  consist of reading the completion report and running the code and
  repeating the reported results.

\item {\bf Month 10: feedback.}  The student will write a
  1-page report describing his results in checking the assigned
  thesis.  The student will meet with his or her partner and give
  verbal feedback on his findings.

\item {\bf Month 11: feedback-response.}  The student will
  write a 2-page report citing the partern's findings and come up with
  a plan to fix the problems.

\item {\bf Month 16: master's-report draft.}  The student will complete the
  work and the master's report.

\item {\bf Month 17: advisor feedback.}  DeepSea team will return to
  the students with feedback about how to proceed.

\item {\bf Month 18: advisor response.} The student will write a
  1-page report with the plan of how to complete.

\item {\bf Month 22: master's code completion.} The student will
  complete all implementation work, checking the work into the
  repository.

\item {\bf Month 24: master's report.} The student will complete the
  master's report.
\end{itemize}


\subsection{Administrivia}

The goal of all the projects will be to complete a report, an
implementation, and a set of theoretical and empirical results that
can be understood and repeated by other people.  Depending on the
results, we will plan to publish a paper on each thesis.  This paper
can be a short 2-4 page paper or a long 10-12 page paper.  All
implementations will be released publicly under the Apache or GNU
public licenses, copyrighted by the student and the DeepSea-Project
Team (Acar, Chargueraud, Rainey).  The papers will likewise be
co-authored by the student(s) and the DeepSea Team.

\subsection{Student selection}

The projects are sorted in the order of their expected difficulty
(more or less). The last three projects on distributed scheduling,
distributed shared memory, and distributed parallel computing are
challenging and should be assigned to students that will have at least
2 years to work on them. Undergraduates can start on them as well and
can write an undergraduate thesis on their initial work and then go on
to continue working on the project for their masters.  It would be
best for these three projects to be assigned to three students that
have worked well together in the past, because if they can work well
together, they can put together their work to create something really
cool, as for example described by the last project, which can act as
an umbrella project for this all.



\subsection{Project PDP: Parallel Dynamic Programming}
Dynamic programming algorithms typically store some intermediate
computations in an array data structure so that they can be used to
speed-up future computations.  They are usually described as
sequential algorithms but appear to be good candidates for
parallelization.  

In this project, the student will develop techniques for parallelizing
existing dynamic programming algorithms or devising new techniques
when serial algorithms resist parallelization.  The algorithms will be
analyzed, implemented, and evaluated.

\subsection{Project PGA: Work-Efficient and Scalable Parallel Graph
  Algorithms}

Graph algorithms are fundamental to many problems.  Unfortunately,
parallel graph algorithms that are competitive with sequential
algorithms on 1 core (processor)---such algorithms are called {\em
  work efficient}---and that also scale well to large numbers of cores
are difficult to design, analyze, and implement.  We need new
techniques in the form of parallel data structures and algorithms for
processing graphs on modern shared memory architectures such as
multicore computers.

In recent work, we have proposed a new data structure called for graph
traversal algorithms so that work-efficiency and scalability can be
guaranteed~\cite{AcarChRa14-graphs}.  The key idea behind that work is
to represent the {\em frontier set} in a graph traversal algorithm,
which consist of the vertices that are about to be visited (but not
yet visited) in such a way to allow them to be partitioned between
multiple cores efficiently without paying undue overheads.

Building on this recent, in this project, the student will develop new
graph algorithm such as the following.  These algorithms can be found
in an online book~\cite{AB-book}.  Some related work include Shun et
al's recent papers in SPAA (for example on connectivity, minimum
spanning trees).

\begin{itemize}
\item
Parallel Bellmon-Ford algorithm for shortest paths.

\item
Parallel graph contraction.

\item
Parallel graph connectivity.

\item
Parallel minimum spanning trees.
\end{itemize}


\subsection{Project PGC: Parallel Granularity Control}

The granularity problem is a longstanding problem in the field of
parallel programming.  At some level, every parallel computation
spawns parallel threads.  Spawning threads imposes some cost on the
run time of the computation.  If the sum of these costs is large,
overheads become excessive, even to the point of negating the benefits
of parallelism.  The challenge is to ensure that the thread-creation
(and destruction) costs are well amortized, under the condition that
performance overall is not harmed in the process.

In our past work, we proposed a granularity-control technique called
Oracle Scheduling.  Our study of Oracle Scheduling consists of an
analytical evaluation and a prototype implementation that we wrote for
the compiler and runtime of a parallel functional
language~\cite{AcarChRa11}.  On the theoretical side, our results show
that, under certain assumptions, we can ensure that all
thread-creation (and destruction) costs are well amortized.  On the
practical side, we showed that our implementation could effectively
control granularity of several small benchmark applications.  Our
study left a a few key questions to future work.  First, in an
imperative parallel language, such as Cilk, what would the
implementation of our technique look like?  Second, if implemented as
a library, could our technique deliver good results, or are compiler
techniques crucial?  In our original study, we considered the
application of our technique to several kernels, such as sorting,
sparse-matrix multiplication, etc.  Third, how would our technique
scale to larger codes?  In summary, answers to these questions will
improve our understanding of the limits to the generality and
scalability of Oracle Scheduling.

It seems likely that there is no silver bullet for solving the
granularity-control problem.  Most likely, the case is that, in the
future, programmers will need to be aware of multiple techniques with
differing tradeoffs.  However, the research literature lacks a
comprehensive study that identifies the relative strengths and
weakness of the various techniques.  One notable alternative to our
Oracle Scheduling is, for instance, that of Lazy Binary Splitting
(LBS)~\cite{lazy-binary-splitting,JFP:8669069}.  This conference
version of the LBS paper contains a nice survey of a number of
granularity-control techniques.  However, the survey lacks comparison
between the loop-based techniques and our Oracle Scheduling.  One
component of this project is to identify the relative strengths and
weaknesses of the techniques and to look for possibilities for
synthesis between, say, LBS and Oracle Scheduling.

In informal discussions, we already considered using such a synthesis
to control granularity for certain types of irregular graph
computations.  Relatedly, we considered whether we could generalize
our Oracle scheduling to control granularity for parallel pipelined
computations~\cite{lee2013fly}.  Each of these lines of work would
involve generalizing our Oracle Scheduling, which is specific to
divide-and-conquer parallel algorithms, to either parallel fixed-point
algorithms (e.g., parallel pseudo DFS~\cite{CongKoKrLeSa08}) or to
parallel wavefront-style (i.e., pipelined) algorithms.  Because these
computations do not fit readily into our formal model, there are open
questions regarding analytical bounds: can we guarantee similar
amortized worst-case bounds for such computations?  Moreover, there
are questions regarding implementation: what would a practical
implementation look like?

\subsection{Project  PDT: Dynamic Parallel Tree Contraction}
Trees are fundamental to many computational problems. When they are
balanced, it is easy to compute various properties of trees in
parallel.  When they are not balanced, however, such computations can
be difficult.

A well-known technique for performing tree computations in parallel is
Miller and Reif's tree contraction algorithm.  In this project, the
student will first develop and implement a work-efficient parallel
algorithm for tree contraction.  The student will then extend the
algorithm and implementation to support insertion and deletion of
edges while updating the contraction efficiently.  Closely related
work include some of our prior work~\cite{AcarBlHaViWo04,AcarBlVi05},
which also provided a serial implementation that can serve as a good
starting point for this research.

\subsection{Project DWS: Distributed Work Stealing}

The work-stealing algorithm is a randomized load-balancing algorithm
that has been proven to work well both in theory and in practice for
scheduling fine-grained parallel programs on parallel architectures
such as modern multicores.  Although work stealing will likely perform
well in distributed systems, where for example, multiple multicore
computers can be connected via a fast network, there has been
relatively little work on understanding its strengths and weaknesses
in the context of modern systems (though there has been prior work in
the earlier days of computer science).  The most closely related work
is probably the Cilk-NOW project from MIT from mid 90's
(e.g.,~\cite{BlumofeLi97}).

This project requires developing a work-stealing algorithms that would
work well on modern distributed systems (for example, multiple
multicores connected with a network), proving some of its properties,
implementing it and making it work well.

The project can be staged into several pieces as follows.
\begin{itemize}
\item Develop a flat algorithm that does not distinguish between local
  and non-local processors.
\item Implement the algorithm on a multicore computer (initial
  implementation)
\item Extend the algorithm to distinguish between local and non-local
  processors.
\item Implement the algorithm on a multicore computer.
\item Implement the algorithm for a distributed system.
\item Evaluate in theory and in practice.
\item Refine the algorithm and its implementation as needed (optimize).

\end{itemize}

\subsection{Project DSM: Distributed Shared Memory}

Distributed shared memory refers to an abstraction that provides a
shared memory view of the world on a disjoint set of computers
connected via a fast network (distributed system).  An efficient
distributed shared memory system can greatly simplify writing programs
that perform large parallel computations by offering a unified view of
memory to the programmer.  Although the need for such systems are
expected to increase, there is relatively little provably and
practically efficient distributed shared memory abstractions and
implementations.

This project requires developing a distributed shared memory
abstraction suitable primarily for parallel fork-join programs.  The
key idea is to match the structure of the memory to that of the
computation by proposing a hierarchial view of memory, where each
thread executing in parallel ``checks out'' versions of the data that
it needs from the main memory by making a copy, modifies them locally,
and checks them back into the main memory, in a way that is similar to
DAG-consistent distributed shared memory~\cite{BFJLR96}.

The project can be staged into several pieces as follows.
\begin{itemize}
\item Develop a flat algorithm that does not distinguish between local
  and non-local processors but implements local per parallel
  ``thread'' memory.
\item Implement the algorithm on a multicore computer (initial
  implementation)
\item Extend the algorithm to distinguish between local and non-local
  processors.
\item Implement the algorithm on a multicore computer.
\item Implement the algorithm for a distributed system.
\item Evaluate in theory and in practice.
\item Refine the algorithm and its implementation as needed (optimize).
\end{itemize}

\subsection{Project DPC: Distributed Parallel Computing}
This project is tightly integrated with the prior two projects.  The
goal is to enable writing distributed parallel programs at a high
level that use the distributed memory and schedulers implemented in
separate projects to execute on multiple multicore computers connected
via a network efficiently and reliably.  The key research component
will be development of techniques for recovering from failures.  For
example, when one of the machines fail, the system should be able to
detect the failure take corrective action. This project also involves
working closely with the other two projects to make sure that they can
be put together in a way so that they both work well together.



{%\def\bibfont{\fontsize{9pt}{10pt}\selectfont}
%\def\bibfont{\fontsize{8pt}{9pt}\selectfont}
%\renewcommand{\baselinestretch}{0.95}
\setlength{\bibsep}{1pt}
% \bibliographystyle{plainnat}  % author-year citations, longer bibliography
% \bibliographystyle{abbrvnat}  % author-year citations,  shorter bibliography (author initials only)
 \bibliographystyle{plain}   % numeric
\bibliography{../../bibliography/main,../../bibliography/gc}
}


\end{document}
